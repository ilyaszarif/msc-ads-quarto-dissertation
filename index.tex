% This is a LaTeX thesis template for the University of Exeter.
% to be used with Quarto
% This template was produced by Rob Hyndman, and formatted by Elliot Gould and
% edited by Muhammad Ilyas
% Updated: 6 June 2023

\documentclass{uniexeterthesis}

%%%%%%%%%%%%%%%%%%%%%%%%%%%%%%%%%%%%%%%%%%%%%%%%%%%%%%%%%%%%%%%
% Add any LaTeX packages and other preamble here if required
%%%%%%%%%%%%%%%%%%%%%%%%%%%%%%%%%%%%%%%%%%%%%%%%%%%%%%%%%%%%%%%

\author{Muhammad Ilyas}
\title{Differntial Gene Expression Modelling}
\def\degreetitle{Master of Science}
\degrees{M.Sc., University of Exeter}
\def\affiliation{Faculty of Environment, Science and Economy}

% Add subject and keywords below
\hypersetup{
     %pdfsubject={The Subject},
     %pdfkeywords={Some Keywords},
     pdfauthor={Muhammad Ilyas},
     pdftitle={Differntial Gene Expression Modelling},
     pdfproducer={Quarto with LaTeX}
}


\bibliography{thesisrefs.bib}

\graphicspath{{figures/}}

\begin{document}

\pagenumbering{roman}

\titlepage

{\setstretch{1.2}\sf\tighttoc\doublespacing}

\bookmarksetup{startatroot}

\hypertarget{copyright-notice}{%
\chapter*{Copyright notice}\label{copyright-notice}}
\addcontentsline{toc}{chapter}{Copyright notice}

\markboth{Copyright notice}{Copyright notice}

© Muhammad Ilyas (2023).

\begin{quote}
Delete following statement if not relevant.
\end{quote}

I certify that I have made all reasonable efforts to secure copyright
permissions for third-party content included in this thesis and have not
knowingly added copyright content to my work without the owner's
permission.

\bookmarksetup{startatroot}

\hypertarget{abstract}{%
\chapter*{Abstract}\label{abstract}}
\addcontentsline{toc}{chapter}{Abstract}

\markboth{Abstract}{Abstract}

The abstract should outline the main approach and findings of the thesis
and must not be more than 500 words.

\bookmarksetup{startatroot}

\hypertarget{declaration}{%
\chapter*{Declaration}\label{declaration}}
\addcontentsline{toc}{chapter}{Declaration}

\markboth{Declaration}{Declaration}

\begin{quote}
Use only one of the following declarations.
\end{quote}

\hypertarget{standard-thesis}{%
\section*{Standard thesis}\label{standard-thesis}}
\addcontentsline{toc}{section}{Standard thesis}

\markright{Standard thesis}

This thesis is an original work of my research and contains no material
which has been accepted for the award of any other degree or diploma at
any university or equivalent institution and that, to the best of my
knowledge and belief, this thesis contains no material previously
published or written by another person, except where due reference is
made in the text of the thesis.

Student name:

Student signature:

Date:

\hypertarget{publications-during-enrolment}{%
\subsection*{Publications during
enrolment}\label{publications-during-enrolment}}
\addcontentsline{toc}{subsection}{Publications during enrolment}

\begin{quote}
Remove this section if you do not have publications.
\end{quote}

The material in Chapter~\ref{sec-intro} has been submitted to the
journal \emph{Journal of Impossible Results} for possible publication.

The contribution in Chapter~\ref{sec-litreview} of this thesis was
presented in the International Symposium on Nonsense held in Dublin,
Ireland, in July 2022.

\hypertarget{reproducibility-statement}{%
\subsection*{Reproducibility
statement}\label{reproducibility-statement}}
\addcontentsline{toc}{subsection}{Reproducibility statement}

This thesis is written using Quarto with renv (Ushey, 2022) to create a
reproducible environment. All materials (including the data sets and
source files) required to reproduce this document can be found at the
Github repository
\href{https://github.com/egouldo/quarto-thesis}{\texttt{github.com/egouldo/quarto-thesis}}.

This work is licensed under a
\href{http://creativecommons.org/licenses/by-nc-sa/4.0/}{Creative
Commons Attribution-NonCommercial-ShareAlike 4.0 International License}.

\hypertarget{thesis-including-published-works-declaration}{%
\section*{Thesis including published works
declaration}\label{thesis-including-published-works-declaration}}
\addcontentsline{toc}{section}{Thesis including published works
declaration}

\markright{Thesis including published works declaration}

I hereby declare that this thesis contains no material which has been
accepted for the award of any other degree or diploma at any university
or equivalent institution and that, to the best of my knowledge and
belief, this thesis contains no material previously published or written
by another person, except where due reference is made in the text of the
thesis.

This thesis includes ?? original papers published in peer reviewed
journals and ?? submitted publications. The ideas, development and
writing up of all the papers in the thesis were the principal
responsibility of myself, the student, working within the Department of
Econometrics \& Business Statistics under the supervision of ??

(The inclusion of co-authors reflects the fact that the work came from
active collaboration between researchers and acknowledges input into
team-based research.)

In the case of (??insert chapter numbers) my contribution to the work
involved the following:

\begin{verbatim}
Warning: package 'tidyverse' was built under R version 4.2.3
\end{verbatim}

\begin{verbatim}
Warning: package 'tibble' was built under R version 4.2.3
\end{verbatim}

\begin{verbatim}
Warning: package 'tidyr' was built under R version 4.2.3
\end{verbatim}

\begin{verbatim}
Warning: package 'readr' was built under R version 4.2.3
\end{verbatim}

\begin{verbatim}
Warning: package 'purrr' was built under R version 4.2.3
\end{verbatim}

\begin{verbatim}
Warning: package 'dplyr' was built under R version 4.2.3
\end{verbatim}

\begin{verbatim}
Warning: package 'forcats' was built under R version 4.2.3
\end{verbatim}

\begin{verbatim}
Warning: package 'knitr' was built under R version 4.2.3
\end{verbatim}

\begin{verbatim}
Warning: package 'kableExtra' was built under R version 4.2.3
\end{verbatim}

\begingroup\fontsize{10}{12}\selectfont

\resizebox{\linewidth}{!}{
\begin{tabu} to \linewidth {>{\raggedleft\arraybackslash}p{1.2cm}>{\raggedright\arraybackslash}p{2.6cm}>{\raggedright}X>{\raggedright\arraybackslash}p{2.6cm}>{\raggedright\arraybackslash}p{2.6cm}>{\raggedright\arraybackslash}p{2.6cm}}
\toprule
\multicolumn{1}{>{\raggedright\arraybackslash}p{1.2cm}}{\textbf{Thesis chapter}} & \multicolumn{1}{>{\raggedright\arraybackslash}p{2.6cm}}{\textbf{Publication title}} & \multicolumn{1}{l}{\textbf{Status}} & \multicolumn{1}{>{\raggedright\arraybackslash}p{2.6cm}}{\textbf{Nature and \% of student contribution}} & \multicolumn{1}{>{\raggedright\arraybackslash}p{2.6cm}}{\textbf{Nature and \% of coauthors' contribution}} & \multicolumn{1}{>{\raggedright\arraybackslash}p{2.6cm}}{\textbf{Coauthors are University of Melbourne students}}\\
\midrule
2 & The life cycle of Mongolian crickets & Submitted & Concept and data analysis, writing first draft: 60\% & Shu Xu, input into manuscript: 25\%; Eddie Betts, input into manuscript: 15\% & Shu Xu: No; Eddie Betts: Yes\\
\bottomrule
\end{tabu}}
\endgroup{}

I have / have not renumbered sections of submitted or published papers
in order to generate a consistent presentation within the thesis.

Student name:

Student signature:

Date:

I hereby certify that the above declaration correctly reflects the
nature and extent of the student's and co-authors' contributions to this
work. In instances where I am not the responsible author I have
consulted with the responsible author to agree on the respective
contributions of the authors.

Main Supervisor name:

Main Supervisor signature:

Date:

\bookmarksetup{startatroot}

\hypertarget{acknowledgements}{%
\chapter*{Acknowledgements}\label{acknowledgements}}
\addcontentsline{toc}{chapter}{Acknowledgements}

\markboth{Acknowledgements}{Acknowledgements}

I would like to thank my pet goldfish for \ldots{}

\begin{quote}
In accordance with Chapter 7.1.4 of the research degrees handbook, if
you have engaged the services of a~professional~editor, you
must~provide~their name~and a brief description of the service rendered.
If the professional editor's current or former area of academic
specialisation is similar your own, this too should be stated as it may
suggest to examiners that the editor's advice to the student has
extended beyond guidance on English expression to affect the substance
and structure of the thesis.
\end{quote}

\begin{quote}
Free text section for you to record your acknowledgment and gratitude
for the more general academic input and support such as financial
support from grants and scholarships and the non-academic support you
have received during the course of your enrolment. If you are a
recipient of the ``Australian Government Research Training Program
Scholarship'', you are required to include the following statement:
\end{quote}

\begin{quote}
\begin{quote}
``This research was supported by an Australian Government Research
Training Program (RTP) Scholarship.''
\end{quote}
\end{quote}

\begin{quote}
You may also wish to acknowledge significant and substantial
contribution made by others to the research, work and writing
represented and/or reported in the thesis. These could include
significant contributions to: the conception and design of the project;
non-routine technical work; analysis and interpretation of research
data; drafting significant parts of the work or critically revising it
so as to contribute to the interpretation.
\end{quote}

\clearpage\pagenumbering{arabic}\setcounter{page}{0}

\bookmarksetup{startatroot}

\hypertarget{sec-intro}{%
\chapter{Introduction}\label{sec-intro}}

This is where you introduce the main ideas of your thesis, and an
overview of the context and background.

In a PhD, Chapter 2 would normally contain a literature review.
Typically, Chapters 3--5 would contain your own contributions. Think of
each of these as potential papers to be submitted to journals. Finally,
Chapter 6 provides some concluding remarks, discussion, ideas for future
research, and so on. Appendixes can contain additional material that
don't fit into any chapters, but that you want to put on record. For
example, additional tables, output, etc.

\hypertarget{quarto}{%
\section{Quarto}\label{quarto}}

In this template, the rest of the chapter shows how to use quarto. The
big advantage of using quarto is that it allows you to include your R or
Python code directly into your thesis, to ensure there are no errors in
copying and pasting, and that everything is reproducible. It also helps
you stay better organized.

For details on using Quarto, see \url{http://quarto.org}.

\hypertarget{data}{%
\section{Data}\label{data}}

Included in this template is a file called \texttt{sales.csv}. This
contains quarterly data on Sales and Advertising budget for a small
company over the period 1981--2005. It also contains the GDP (gross
domestic product) over the same period. All series have been adjusted
for inflation. We can load in this data set using the following code:

\begin{Shaded}
\begin{Highlighting}[]
\NormalTok{sales }\OtherTok{\textless{}{-}}\NormalTok{ readr}\SpecialCharTok{::}\FunctionTok{read\_csv}\NormalTok{(here}\SpecialCharTok{::}\FunctionTok{here}\NormalTok{(}\StringTok{"data/sales.csv"}\NormalTok{)) }\SpecialCharTok{\%\textgreater{}\%}
  \FunctionTok{rename}\NormalTok{(}\AttributeTok{Quarter =} \StringTok{\textasciigrave{}}\AttributeTok{...1}\StringTok{\textasciigrave{}}\NormalTok{) }\SpecialCharTok{\%\textgreater{}\%}
  \FunctionTok{mutate}\NormalTok{(}
    \AttributeTok{Quarter =} \FunctionTok{as.Date}\NormalTok{(}\FunctionTok{paste0}\NormalTok{(}\StringTok{"01{-}"}\NormalTok{, Quarter), }\StringTok{"\%d{-}\%b{-}\%y"}\NormalTok{),}
    \AttributeTok{Quarter =} \FunctionTok{yearquarter}\NormalTok{(Quarter)}
\NormalTok{  ) }\SpecialCharTok{\%\textgreater{}\%}
  \FunctionTok{as\_tsibble}\NormalTok{(}\AttributeTok{index =}\NormalTok{ Quarter)}
\end{Highlighting}
\end{Shaded}

\begin{verbatim}
New names:
Rows: 100 Columns: 4
-- Column specification
-------------------------------------------------------- Delimiter: "," chr
(1): ...1 dbl (3): Sales, AdBudget, GDP
i Use `spec()` to retrieve the full column specification for this data. i
Specify the column types or set `show_col_types = FALSE` to quiet this message.
* `` -> `...1`
\end{verbatim}

Any data you use in your thesis can go into the \texttt{data} directory.
The data should be in exactly the format you obtained it. Do no editing
or manipulation of the data prior to including it in the \texttt{data}
directory. Any data munging should be scripted and form part of your
thesis files (possibly hidden in the output).

\hypertarget{figures}{%
\section{Figures}\label{figures}}

Figure~\ref{fig-deaths} shows time plots of the data we just loaded.
Notice how figure captions and references work. Chunk names can be used
as figure labels with \texttt{fig-} prefixed. Never manually type figure
numbers, as they can change when you add or delete figures. This way,
the figure numbering is always correct.

\begin{figure}

{\centering \includegraphics{01-chap1_files/figure-pdf/fig-deaths-1.pdf}

}

\caption{\label{fig-deaths}Quarterly sales, advertising and GDP data.}

\end{figure}

\hypertarget{results-from-analyses}{%
\section{Results from analyses}\label{results-from-analyses}}

We can fit a dynamic regression model to the sales data.

\begin{verbatim}
Series: Sales
Model: LM w/ ARIMA(1,0,0)(0,1,1)[4] errors

Coefficients:
         ar1     sma1     GDP  AdBudget
      0.2189  -0.9016  0.9742    2.2824
s.e.  0.1022   0.0715  0.4387    0.3930

sigma^2 estimated as 1677:  log likelihood=-493.94
AIC=997.87   AICc=998.54   BIC=1010.69
\end{verbatim}

If \(y_t\) denotes the sales in quarter \(t\), \(x_t\) denotes the
corresponding advertising budget and \(z_t\) denotes the GDP, then the
resulting model is: \begin{equation}\protect\hypertarget{eq-drm}{}{
  y_t - y_{t-4} = \beta (x_t-x_{t-4}) + \gamma (z_t-z_{t-4}) + \phi_1 (y_{t-1} - y_{t-5}) + \Theta_1 \varepsilon_{t-4} + \varepsilon_t
}\label{eq-drm}\end{equation} where \(\beta = 2.28\), \(\gamma = 0.97\),
\(\phi_1 = 0.22\), and \(\Theta_1 = -0.90\). We can reference this
equation using Equation~\ref{eq-drm}.

\hypertarget{tables}{%
\section{Tables}\label{tables}}

Let's assume future advertising spend and GDP are at the current levels.
Then forecasts for the next year are given in
Table~\ref{tbl-salesforecasts}.

\hypertarget{tbl-salesforecasts}{}
\begin{longtable}[]{@{}lr@{}}
\caption{\label{tbl-salesforecasts}Forecasts for the next year assuming
Advertising budget and GDP are unchanged.}\tabularnewline
\toprule\noalign{}
Quarter & Sales forecast \\
\midrule\noalign{}
\endfirsthead
\toprule\noalign{}
Quarter & Sales forecast \\
\midrule\noalign{}
\endhead
\bottomrule\noalign{}
\endlastfoot
2006 Q1 & 1000.2 \\
2006 Q2 & 1013.1 \\
2006 Q3 & 1076.7 \\
2006 Q4 & 1003.5 \\
\end{longtable}

Again, notice the use of labels and references to automatically generate
table numbers.

\bookmarksetup{startatroot}

\hypertarget{sec-litreview}{%
\chapter{Literature Review}\label{sec-litreview}}

This chapter contains a summary of the context in which your research is
set.

Imagine you are writing for your fellow PhD students. Topics that are
well-known to them do not have to be included here. But things that they
may not know about should be included.

Resist the temptation to discuss everything you've read in the last few
years. And you are not writing a textbook either. This chapter is meant
to provide the background necessary to understand the material in
subsequent chapters. Stick to that.

You will need to organize the literature review around themes, and
within each theme provide a story explaining the development of ideas to
date. In each theme, you should get to the point where your ideas will
fit in. But leave your ideas to later chapters. This way it is clear
what has been done beforehand, and what new contributions you are making
to the research field.

All citations should be done using markdown notation as shown below.
This way, your bibliography will be compiled automatically and
correctly.

\hypertarget{sec-expsmooth}{%
\section{Exponential smoothing}\label{sec-expsmooth}}

Exponential smoothing methods were originally developed in the late
1950s (Brown, 1959, 1963; Holt, 1957; Winters, 1960). Because of their
computational simplicity and interpretability, they became widely used
in practice.

Empirical studies by Makridakis \& Hibon (1979) and Makridakis et al.
(1982) found little difference in forecast accuracy between exponential
smoothing and ARIMA models. This made the family of exponential
smoothing procedures an attractive proposition (see Chatfield et al.,
2001).

The methods were less popular in academic circles until Ord et al.
(1997) introduced a state space formulation of some of the methods,
which was extended in Hyndman et al. (2002) to cover the full range of
exponential smoothing methods.

\bookmarksetup{startatroot}

\hypertarget{bibliography}{%
\chapter*{Bibliography}\label{bibliography}}
\addcontentsline{toc}{chapter}{Bibliography}

\markboth{Bibliography}{Bibliography}

\hypertarget{refs}{}
\begin{CSLReferences}{1}{0}
\leavevmode\vadjust pre{\hypertarget{ref-Brown59}{}}%
Brown, R. G. (1959). \emph{Statistical forecasting for inventory
control}. McGraw-Hill, New York.

\leavevmode\vadjust pre{\hypertarget{ref-Brown63}{}}%
Brown, R. G. (1963). \emph{Smoothing, forecasting and prediction of
discrete time series}. Englewood Cliffs, New Jersey: Prentice Hall.

\leavevmode\vadjust pre{\hypertarget{ref-CKOS01}{}}%
Chatfield, C., Koehler, A. B., Ord, J. K., \& Snyder, R. D. (2001). A
new look at models for exponential smoothing. \emph{The Statistician},
\emph{50}(2), 147--159.

\leavevmode\vadjust pre{\hypertarget{ref-Holt57}{}}%
Holt, C. E. (1957). \emph{Forecasting trends and seasonals by
exponentially weighted averages} (O.N.R. Memorandum No. 52/1957).
Carnegie Institute of Technology.

\leavevmode\vadjust pre{\hypertarget{ref-HKSG02}{}}%
Hyndman, R. J., Koehler, A. B., Snyder, R. D., \& Grose, S. (2002). A
state space framework for automatic forecasting using exponential
smoothing methods. \emph{International Journal of Forecasting},
\emph{18}(3), 439--454.

\leavevmode\vadjust pre{\hypertarget{ref-M1comp}{}}%
Makridakis, S., Anderson, A., Carbone, R., Fildes, R., Hibon, M.,
Newton, R. L. J., Parzen, E., \& Winkler, R. (1982). The accuracy of
extrapolation (time series) methods: Results of a forecasting
competition. \emph{Journal of Forecasting}, \emph{1}, 111--153.

\leavevmode\vadjust pre{\hypertarget{ref-MH79}{}}%
Makridakis, S., \& Hibon, M. (1979). Accuracy of forecasting: An
empirical investigation (with discussion). \emph{Journal of Royal
Statistical Society (A)}, \emph{142}, 97--145.

\leavevmode\vadjust pre{\hypertarget{ref-OKS97}{}}%
Ord, J. K., Koehler, A. B., \& Snyder, R. D. (1997). Estimation and
prediction for a class of dynamic nonlinear statistical models.
\emph{Journal of American Statistical Association}, \emph{92},
1621--1629.

\leavevmode\vadjust pre{\hypertarget{ref-renv}{}}%
Ushey, K. (2022). \emph{{renv}: Project environments}.
\url{https://CRAN.R-project.org/package=renv} R package version 0.16.0

\leavevmode\vadjust pre{\hypertarget{ref-Winters60}{}}%
Winters, P. R. (1960). Forecasting sales by exponentially weighted
moving averages. \emph{Management Science}, \emph{6}, 324--342.

\end{CSLReferences}

\cleardoublepage
\phantomsection
\addcontentsline{toc}{part}{Appendices}
\appendix

\hypertarget{additional-stuff}{%
\chapter{Additional stuff}\label{additional-stuff}}

You might put some computer output here, or maybe additional tables. It
is possible to have multiple appendices. Just list them in the
appropriate place within \texttt{\_quarto.yml}.


\end{document}
